%
%  untitled
%
%  Created by Dean Freestone on 2010-07-13.
%  Copyright (c) 2010 . All rights reserved.
%
\documentclass[]{article}

% Use utf-8 encoding for foreign characters
\usepackage[utf8]{inputenc}

% Setup for fullpage use
\usepackage{fullpage}

% Uncomment some of the following if you use the features
%
% Running Headers and footers
%\usepackage{fancyhdr}

% Multipart figures
%\usepackage{subfigure}

% More symbols
%\usepackage{amsmath}
%\usepackage{amssymb}
%\usepackage{latexsym}

% Surround parts of graphics with box
\usepackage{boxedminipage}

% Package for including code in the document
\usepackage{listings}

% If you want to generate a toc for each chapter (use with book)
\usepackage{minitoc}

% This is now the recommended way for checking for PDFLaTeX:
\usepackage{ifpdf}

%\newif\ifpdf
%\ifx\pdfoutput\undefined
%\pdffalse % we are not running PDFLaTeX
%\else
%\pdfoutput=1 % we are running PDFLaTeX
%\pdftrue
%\fi

\ifpdf
\usepackage[pdftex]{graphicx}
\else
\usepackage{graphicx}
\fi
\title{Brain Stimulation Induced Interactions}
\author{In no particular order: Dean R. Freestone, Tim S. Nelson, Alan Lai, Simon Vogrin,\\
 Michael Murphy?, David B. Grayden, Mark J. Cook}

\date{2010-07-13}

\begin{document}

\ifpdf
\DeclareGraphicsExtensions{.pdf, .jpg, .tif}
\else
\DeclareGraphicsExtensions{.eps, .jpg}
\fi

\maketitle


\begin{abstract}
\end{abstract}

\section{Introduction}
\subsection{From LEW Carty Application}
Epilepsy is a chronic disease of the brain that affects 1-2\% of people. Its defining characteristic is recurrent seizures, carrying a risk of injury, brain damage or death. Anti-epileptic drugs are the mainstay of treatment.  Despite this, one third of focal epilepsy patients have uncontrolled seizures. For these people, seizure onset is unpredictable, severely impairing quality of life.

Although seizure occurrence appears to be random, there is evidence that the brain undergoes subtle changes prior to seizures. This evidence was initially anecdotal, with patients and their family and friends reporting strange feelings or behaviour in the minutes or hours prior to seizures. More recently, scientists have observed changes in the form of hyper-activity in the brain’s dynamics prior to seizures, using a variety of medical imaging techniques.

The field of seizure prediction has developed considerably over the last 30 years, yet today this important problem is still unsolved....

\subsection{From NHMRC Application}
Epilepsy is a chronic disease of the brain that affects around 1-2\% of the world’s population~\cite{Beran1985,Dua2006}. The high incidence of epilepsy is due to the large number of possible causes including developmental abnormalities, genetic abnormalities, febrile convulsions and brain insults, such as head trauma, hypoxia, ischemia, tumours and central nervous system infections. Head trauma remains a major cause, and is often particularly difficult to treat. However, approximately one third of epilepsies are cryptogenic~\cite{Theodore2006}.
The defining characteristic of epilepsy is recurrent seizures, which reflect a sudden excess of hyper-synchronous activity of neurons in the cerebral cortex. EEG recordings of epileptic seizures show that these hyper-synchronous discharges may begin locally in one hemisphere; these are called partial or focal seizures, with single or multiple foci in portions of one cerebral hemisphere. Seizures may also occur simultaneously in both cerebral hemispheres; these are called generalised seizures. Partial seizures may remain localised with sensory, motor, cognitive, or autonomic symptoms, leading to loss of awareness or more severe symptoms associated with generalised convulsive seizures. Generalised seizures cause altered consciousness at their onset, followed by motor symptoms of varying severity, ranging from brief body jerks to generalised tonic-clonic (convulsive) movements. Seizures may occur from hundreds of times per day to once every few years. Frequent or lengthy uncontrollable seizures carry a risk of irreversible brain damage, and the syndrome of sudden unexpected death (SUDEP) is a common cause of death of people with epilepsy.
Anti-epileptic drugs (AEDs) are currently the mainstay of epilepsy treatment. Despite state-of-the-art medical management with modern AEDs, at least 30\% of patients with focal epilepsy continue to have frequent life-threatening seizures~\cite{Schmidt2005}. For patients who do not respond to AEDs, a possible treatment is surgical resection of pathological brain tissue.  However, surgery is not a viable option for the great majority of this patient group, because the responsible lesions are too large, multiple, cannot be defined, or in eloquent brain regions. For these people, seizures may strike at any time, leaving them severely restricted in their day-to-day activities. The hazard associated with seizures is related to the convulsive episode itself to some extent, but the major hazard relates to the circumstances of the event – including driving, swimming, and operating dangerous machinery. The most significant factor leading to disability for these people is the current inability to anticipate and control seizures. This places enormous limitations upon patients in virtually every sphere of their lives, by impacting upon family, social, educational, and vocational activities. Everyday activities such as driving a car or playing sport are at best fraught with insecurity and at worst become impossible.
Although seizure occurrences appear to be random, there is evidence that changes occur in the brain’s dynamical behaviour prior to attacks. Partly, this evidence is provided by anecdotal reports of prodromes (i.e., subtle changes in behaviour) from sufferers and their carers in the hours or days before a seizure occurs. In addition, imaging studies have shown metabolic levels increase immediately prior to seizures~\cite{Zhao2007}. Also, transcranial magnetic stimulation experiments have shown the brain is in a hyper-excitable state prior to seizures~\cite{Badawy2009,Wright2006}. This evidence suggests that seizures may be anticipated by tracking the excitability levels and dynamics of the brain. The reliable anticipation of seizures will enable administration of a focal therapy, such as electrical stimulation or drug delivery, thereby reducing the severity of the impending seizure or, ideally, eliminating it altogether. 

\subsubsection{Conventional seizure anticipation methods are not clinically useful}
Predicting seizure occurrence is a challenging problem that remains unsolved after 30 years of intensive research. The majority of previous approaches have used ongoing EEG recordings (passively observed) to extract features describing the current ‘state’ of the brain. Some of these algorithms involved estimating entropy, correlation dimension, and short-term Lyapunov exponents~\cite{Babloyantz1986,Pijn1991,Pritchard1995,Iasemidis1996,LeVanQuyen2001}. 
The research focus in EEG-based seizure anticipation has since shifted to iEEG synchronisation analysis after the aforementioned methods failed to deliver repeatable results~\cite{Lai2003,McSharry2003,Maiwald2004,Lai2004}. Synchronisation measures are thought to be correlates of cortical excitability reflecting the likelihood of seizure occurrence~\cite{Kalitzin2002}. Although these algorithms have shown promise in certain patient groups, they have not delivered reproducible outcomes and therefore have not provided satisfactory clinical performance~\cite{Lehnertz2007,Mormann2007}. 
Although the aforementioned methods are mathematically quite different, they are all conceptually similar and focused on measuring the degree of order within the brain, where a decrease in complexity indicates an abnormal hyper-synchronous state associated with a pre-seizure state. 

\subsubsection{Active monitoring of excitability holds significant promise for seizure anticipation}
The active probing approach represents a significant paradigm shift in the field of seizure prediction and has the potential to solve this important problem. Recent theoretical evidence suggests an active approach is required for seizure anticipation~\cite{Suffczynski2008}. This supports our methodology in the proposed study. In addition, our colleagues in the Dept of Electrical and Electronic Engineering at the University of Melbourne have provided further evidence for the need of an active approach. It was shown that the failure of EEG-based seizure prediction algorithms is not because the correct EEG feature has not been found, but rather that seizure precursors are not observable from the passively recorded EEG signal itself.  That is, the passive EEG only allows the observation of a very small fraction of the underlying generators of brain activity (O’Sullivan-Greene et al. 2010, O’Sullivan-Greene et al. 2009a). This indicates that seizure anticipation from passive EEG is unlikely to succeed. However, it was shown (through computer simulation) that the clever use of a probing stimulus can extract information from the EEG to facilitate seizure anticipation, even if the underlying neuro-dynamics can not be observed in the passive EEG measurements (O’Sullivan-Greene et al. 2009b). 
Our results suggest that the lack of information in the passive EEG signal can be overcome through active electrical probing for two reasons. First, electrical probing is a way to raise EEG signals above the background noise level by averaging over multiple trials. Indeed, averaging the direct response to a known stimulus pulse to estimate excitability effectively cancels spurious fluctuations from ongoing brain processes, and provides spatially specific information when stimuli are targeted using intracranial grid electrodes. Second, we expect the brain’s response to be specific to a specific probing signal.  So whilst the EEG only lets us see a small fraction of the underlying dynamics, we can tailor our analysis to ensure that the fraction of dynamics visible in the EEG gives us the specific information necessary for seizure anticipation.
We hypothesize that larger cortical responses to stimulation indicate higher cortical excitability and a greater chance of a seizure occurring. Our preliminary data suggest that the method is effective.
The oscillatory nature of the rhythms of the brain lends them naturally to synchronisation analysis. 
Kalitzin et al. (2005) derived the relative phase clustering index (rPCI), which measures the amount of power in the brain’s rhythmic activity in response to bursts of stimulation at harmonics of the stimulation frequency (they used 10, 15 and 20 biphasic pulses per second in each burst) relative to the power at the stimulation frequency. A high rPCI indicates that the stimulation has entrained the brain’s rhythms, implying a high level of excitability. Their results demonstrated a relationship between the ictal state and elevated excitability levels for mesial temporal lobe epilepsies (MTLE). These results illustrate the feasibility of using a probing stimulus for tracking cortical excitability.

\section{Method}

\subsection{Data}
Data was collected from patients undergoing  evaluation for the surgical resection of epileptic foci at St. Vincent's Public and Private Hospital's in Melbourne, Australia. Data was collected with informed consent from patients under ethics approval from St. Vincent's Public and Private Human Research Ethics Committee (HREC-A 006-08).

\begin{tabular}{c|c|c|c|c} \hline
\textbf{Patient Number (ID)} & \textbf{Electrode Type} & \textbf{Stimulation Channels} & \textbf{Number of Seizures} & Number of Probe Sessions\\ \hline
\end{tabular}


\bibliographystyle{plain}
\bibliography{BrainSII_biblography.bib}
\end{document}
