%
%  untitled
%
%  Created by Dean Freestone on 2010-07-13.
%  Copyright (c) 2010 . All rights reserved.
%

\documentclass[]{article}

\usepackage{color}
\newcommand{\dean}[1]{\textsf{\emph{\textbf{\textcolor{green}{#1}}}}} 
% Use utf-8 encoding for foreign characters
\usepackage[utf8]{inputenc}

% Setup for fullpage use
\usepackage{fullpage}

% Uncomment some of the following if you use the features
%
% Running Headers and footers
%\usepackage{fancyhdr}

% Multipart figures
%\usepackage{subfigure}

% More symbols
\usepackage{amsmath}
\usepackage{amssymb}
%\usepackage{latexsym}

% Surround parts of graphics with box
\usepackage{boxedminipage}

% Package for including code in the document
\usepackage{listings}

% If you want to generate a toc for each chapter (use with book)
\usepackage{minitoc}


% This is now the recommended way for checking for PDFLaTeX:
\usepackage{ifpdf}

%\newif\ifpdf
%\ifx\pdfoutput\undefined
%\pdffalse % we are not running PDFLaTeX
%\else
%\pdfoutput=1 % we are running PDFLaTeX
%\pdftrue
%\fi

\ifpdf
\usepackage[pdftex]{graphicx}
\else
\usepackage{graphicx}
\fi
\title{Brain Stimulation Induced Interactions}
\author{In no particular order: Dean R. Freestone, Tim S. Nelson, Alan Lai, Simon Vogrin,\\
 Michael Murphy?, David B. Grayden, Anthony N. Burkitt, Mark J. Cook}

\date{2010-07-13}

\begin{document}

\ifpdf
\DeclareGraphicsExtensions{.pdf, .jpg, .tif}
\else
\DeclareGraphicsExtensions{.eps, .jpg}
\fi

\maketitle


\begin{abstract}
\end{abstract}

\section{Introduction}
% \subsection{From LEW Carty Application}
% Epilepsy is a chronic disease of the brain that affects 1-2\% of people. Its defining characteristic is recurrent seizures, carrying a risk of injury, brain damage or death. Anti-epileptic drugs are the mainstay of treatment.  Despite this, one third of focal epilepsy patients have uncontrolled seizures. For these people, seizure onset is unpredictable, severely impairing quality of life.
% 
% Although seizure occurrence appears to be random, there is evidence that the brain undergoes subtle changes prior to seizures. This evidence was initially anecdotal, with patients and their family and friends reporting strange feelings or behaviour in the minutes or hours prior to seizures. More recently, scientists have observed changes in the form of hyper-activity in the brain’s dynamics prior to seizures, using a variety of medical imaging techniques.
% 
% The field of seizure prediction has developed considerably over the last 30 years, yet today this important problem is still unsolved....

\subsection{Epilepsy Background}
Epilepsy is a chronic disease of the brain that affects around 1-2\% of the world’s population~\cite{Beran1985,Dua2006}. The high incidence of epilepsy is due to the large number of possible causes including developmental abnormalities, genetic abnormalities, febrile convulsions and brain insults, such as head trauma, hypoxia, ischemia, tumours and central nervous system infections. Head trauma remains a major cause, and is often particularly difficult to treat. However, approximately one third of epilepsies are cryptogenic~\cite{Theodore2006}.

The defining characteristic of epilepsy is recurrent seizures, which reflect a sudden excess of hyper-synchronous activity of neurons in the cerebral cortex. EEG recordings of epileptic seizures show that these hyper-synchronous discharges may begin locally in one hemisphere; these are called partial or focal seizures, with single or multiple foci in portions of one cerebral hemisphere. Seizures may also occur simultaneously in both cerebral hemispheres; these are called generalised seizures. Partial seizures may remain localised with sensory, motor, cognitive, or autonomic symptoms, leading to loss of awareness or more severe symptoms associated with generalised convulsive seizures. Generalised seizures cause altered consciousness at their onset, followed by motor symptoms of varying severity, ranging from brief body jerks to generalised tonic-clonic (convulsive) movements. Seizures may occur from hundreds of times per day to once every few years. Frequent or lengthy uncontrollable seizures carry a risk of irreversible brain damage, and the syndrome of sudden unexpected death (SUDEP) is a common cause of death of people with epilepsy.

Anti-epileptic drugs (AEDs) are currently the mainstay of epilepsy treatment. Despite state-of-the-art medical management with modern AEDs, at least 30\% of patients with focal epilepsy continue to have frequent life-threatening seizures~\cite{Schmidt2005}. For patients who do not respond to AEDs, a possible treatment is surgical resection of pathological brain tissue.  However, surgery is not a viable option for the great majority of this patient group, because the responsible lesions are too large, multiple, cannot be defined, or in eloquent brain regions. For these people, seizures may strike at any time, leaving them severely restricted in their day-to-day activities. The hazard associated with seizures is related to the convulsive episode itself to some extent, but the major hazard relates to the circumstances of the event – including driving, swimming, and operating dangerous machinery. The most significant factor leading to disability for these people is the current inability to anticipate and control seizures. This places enormous limitations upon patients in virtually every sphere of their lives, by impacting upon family, social, educational, and vocational activities. Everyday activities such as driving a car or playing sport are at best fraught with insecurity and at worst become impossible.

Although seizure occurrences appear to be random, there is evidence that changes occur in the brain’s dynamical behaviour prior to attacks. Partly, this evidence is provided by anecdotal reports of prodromes (i.e., subtle changes in behaviour) from sufferers and their carers in the hours or days before a seizure occurs. In addition, imaging studies have shown metabolic levels increase immediately prior to seizures~\cite{Zhao2007}. Also, transcranial magnetic stimulation experiments have shown the brain is in a hyper-excitable state prior to seizures~\cite{Badawy2009,Wright2006}. This evidence suggests that seizures may be anticipated by tracking the excitability levels and dynamics of the brain. The reliable anticipation of seizures will enable administration of a focal therapy, such as electrical stimulation or drug delivery, thereby reducing the severity of the impending seizure or, ideally, eliminating it altogether. 

\subsubsection{Conventional seizure anticipation methods are not clinically useful}
\dean{Will probably start the paper from here.}
Predicting seizure occurrence is a challenging problem that remains unsolved after 30 years of intensive research. The majority of previous approaches have used ongoing EEG recordings (passively observed) to extract features describing the current ‘state’ of the brain. Some of these algorithms involved estimating entropy, correlation dimension, and short-term Lyapunov exponents~\cite{Babloyantz1986, Pijn1991, Pritchard1995, Iasemidis1996, LeVanQuyen2001}. 

The research focus in EEG-based seizure anticipation has since shifted to iEEG synchronisation analysis after the aforementioned methods failed to deliver repeatable results~\cite{Lai2003, McSharry2003, Maiwald2004, Lai2004}. Synchronisation measures are thought to be correlates of cortical excitability reflecting the likelihood of seizure occurrence~\cite{Kalitzin2002}. Although these algorithms have shown promise in certain patient groups, they have not delivered reproducible outcomes and therefore have not provided satisfactory clinical performance~\cite{Lehnertz2007, Mormann2007}. 
Although the aforementioned methods are mathematically quite different, they are all conceptually similar and focused on measuring the degree of complexity within the brain, where a decrease in complexity indicates abnormal hyper-synchronous activity associated with a pre-seizure state. 

\subsubsection{Active monitoring of excitability holds significant promise for seizure anticipation}
The active probing approach represents a significant paradigm shift in the field of seizure prediction and has the potential to solve this important problem. Recent theoretical evidence suggests an active approach is required for seizure anticipation~\cite{Suffczynski2008}. This supports our methodology in the proposed study. Further evidence is provided from a recent study were it was shown that the failure of EEG-based seizure prediction algorithms is not because the correct EEG feature has not been found, but rather that seizure precursors are not observable from passively recorded EEG signals. That is, the passive EEG only allows the observation of a very small fraction of the underlying generators of brain activity~\cite{O'Sullivan-Greene2009, O'Sullivan-Greene2009a}. This indicates that seizure anticipation from passive EEG is unlikely to succeed. However, it was shown (through computer simulation) that the clever use of a probing stimulus can extract information from the EEG to facilitate seizure anticipation, even if the underlying neuro-dynamics can not be observed in the passive EEG measurements~\cite{O'Sullivan-Greene2009a}. 

Kalitzin et al.~\cite{Kalitzin2005} derived the relative phase clustering index (rPCI), which measures the amount of power in the brain’s entrainment at harmonics to bursts of stimulation  of the stimulation frequency (they used 10, 15 and 20 biphasic pulses per second in each burst) relative to the power at the stimulation frequency. A high rPCI indicates that the stimulation has entrained the brain’s rhythms, implying a high level of excitability. Their results demonstrated a relationship between the ictal state and elevated excitability levels for mesial temporal lobe epilepsies (MTLE). These results illustrate the feasibility of using a probing stimulus for tracking cortical excitability.

\section{Method}
\subsection{Data Acquisition and Stimulation Protocol}
Data was collected from patients undergoing evaluation for the surgical resection of epileptic foci at St. Vincent's Public and Private Hospital's in Melbourne, Australia. Data was collected with informed consent under ethics approval from St. Vincent's Public and Private Human Research Ethics Committee (HREC-A 006/08).

The standard clinical practice for epilepsy related surgery involves a diagnostic period of one week, where intracranial EEG electrodes are implanted directly on the surface of the brain. These electrodes are used to map pathological brain tissue by recording abnormal oscillations in the electrical fields of the brain during seizures. In addition the electrodes facilitate mapping of important functional brain tissue, such as speech and motor control, by applying electrical stimulation and observing the behavioural response.  

\begin{tabular}{|p{2.4cm}|p{2.4cm}|p{2.4cm}|p{2.4cm}|p{3cm}|} \hline
\textbf{Patient Number (ID)} & \textbf{Electrode Type} & \textbf{Stimulation Channels} & \textbf{Number of Seizures} & \textbf{Number of Probe Sessions}\\ \hline
1 (AGAH) & 64 channel grid & R Y B W & 1 & 1 \\ \hline
2 (MSOC) & grid & R Y B W & 1 & 1 (subclinical) \\ \hline
3 (OKEM) & Depth & R Y B W & 1 & 3 \\ \hline
4 (IGLC) & Grid & R Y B W & 3 & 1 \\ \hline
5 (RCAS) & Grid & R Y B W & 1 & 1 \\ \hline
6 (UMIK) & Grid & R Y B W & 1 & 1 \\ \hline
7 (EKET) & Grid & R Y B W & 1 & 1 \\ \hline
\end{tabular}

In parallel to the standard clinical procedure, the electrical stimulation protocol was carried out to measure stimulus-induced cortical interactions. The electrical stimuli (delivered by a Grass S88x neurostimulator, Astro-Med) consisted of single bipolar, biphasic pulses delivered in groups of 100, with each pulse separated by 3.01 s. A rest period of 5 minutes occurred between each stimulation group. The pulse width was 100~$\mu$s with current intensity of 1~mA. The current intensity yielded a charge/phase of 0.1~$\mu$C/phase (electrode diameter of 4~mm) and a charge density of approximately 0.2~$\mu$C/phase/cm2. This charge density is two orders of magnitude below the well-documented safety limits of 30~$\mu$C/phase/cm2 (Vonck et al. 2002; Kerrigan et al. 2004; Kuncel \& Grill 2004; Osorio et al. 2007; Velasco et al. 2007). Stimuli were targeted to low-impedance electrodes, away from the cerebral vasculature, to the suspected epileptic foci and surrounding tissue.

All data was collected via intracranial grid electrodes (Ad-Tech medical), which were comprised of an array of platinum disk electrodes with regular spacing of 10~mm and 4~mm diameter. The number of electrodes varied in each patient depending on the clinical needs. Intracranial EEG was sampled at a rate of 5~kHz and the system (Synamp2, Compumedics) allowed for recording from a DC level, so transient responses to stimuli could be observed.

\subsection{Data Analysis}
\subsubsection{Preprocessing}
The first pre-processing step was to identify channels that had very poor signal quality via visual inspection. Data from these channels were set to the $nan$ data type and excluded from further analysis. In the next step, the data was re-referenced to a differential montage to remove common mode artifact and the effect of the reference electrode. Next the data was low-pass filtered with a cut-off frequency of 100~Hz. The data was then resampled from 5~kHz to 1~kHz. To enable spatial visualisation and spatial frequency analysis, the corrupted channels, set to $nan$, were interpolated using all neighbouring channels.

\begin{figure}[htbp]
	\centering
		\includegraphics[scale=0.8]{./Figures/DifferentialMontage.pdf}
	\caption{caption}
	\label{fig:DifMontage}
\end{figure}

Electric potential measured from iEEG has
an oscillatory behavior. By assuming the oscillations are generated
by individual cortical columns, it is possible to study their
interaction. As with many systems in nature, synchronisation
phenomena can be observed. Abnormal levels of phase and frequency
entrainment have been associated with different neurological
disorders, particularly Parkinson's disease and epilepsy.
Synchronisation also seems to be a central mechanism for neural
information processing within a brain, as well as for communication
between different brain areas~\cite{Rosenblum2001}. This concept
was discussed earlier in reference to seizure prediction. Following
the hypothesis that widespread neural interaction is required for
information processing and executing tasks, it makes sense to
quantifying the interactions for both epileptic seizure prediction
and brain computer interfacing. Below a description of some methods
used to observe these events is described.\\

The measurement of synchrony in the brain is a non-trivial task.
Intracranial EEG oscillations are irregular, noisy, non-stationary
and at times chaotic. Therefore any synchrony can be masked. The
problem can be stated as: we are trying to measure interaction
between complex systems with an unknown underlying structure via
data from two time series. This interaction is some kind of
dependance of one system on the other.\\


\subsection{Linear Cross-correlation}
This is the simplest measure of synchronisation, it is defined in
the time domain as a function of the time lag $\tau$, where $\tau =
-(N-1),...,0,...,N-1$ by
\begin{equation}
C_{XY}{\tau} = \begin{array}{l l}
  \frac{1}{N-\tau}\sum_{n=1}^{N-\tau}x_{n+\tau}y_n & \quad \tau \geq 0\\
  C_{YX}(-\tau) & \quad \tau <0\\ \end{array}
\end{equation}

A quantitative measure of this linear correlation is

\begin{equation}
C_{max} = max\{|C_{XY}(\tau)|\}
\end{equation}

This analysis method has the advantage of being very simple, but it
fails to capture the nonlinear dynamics such as phase
synchronisation.

\subsection{Phase Synchronisation}

When quantifying the synchronisation of phases of two times series,
$x$ and $y$, we first need to find the instantaneous phases,
$\phi_x(t)$ and $\phi_y(t)$. Phase locking is defined as
\begin{equation}
m\phi_x(t)-n\phi_y(t) = const
\end{equation}

Where $m$ and $n$ are integers~\cite{Huygens1673}. Phase
entrainment is defined as
\begin{equation}
m\phi_x(t)-n\phi_y(t) < const
\end{equation}

This definition is for the case of noisy and/or chaotic
systems~\cite{Rosenblum1996} such as iEEG.\\

There are multiple ways of extracting the instantaneous phase of a
time series, here we discus two approaches: the analytic signal
approach using the Hilbert transform (HT), and using the
continuous wavelet transform (CWT).\\

Using the analytic signal approach, the analytic signal is defined
as
\begin{equation}
Z_x(t) = x(t) + i\tilde{x}(t) = A_x^H(t)e^{i\phi_x^H(t)}
\end{equation}

where $\tilde{x}(t)$ is the HT of $x(t)$ and

\begin{equation}
\tilde{x}(t) =
\frac{1}{\pi}P.V.\int_{-\infty}^{+\infty}\frac{x(t')}{t-t'}dt'
\end{equation}

where $P.V.$ denotes the Cauchy principle value.

The Hilbert transform can also be calculated using Fourier theory.
\begin{equation}
\tilde{x}(t) = -iFT^{-1}[FT[x(t)]sign(\omega)]
\end{equation}

This definition demonstrates that the transform gives a translation
of the phase in the frequency domain by $\frac{\pi}{2}$, for all
frequencies, whilst the power remains unchanged.\\

The Hilbert phase is defined as
\begin{equation}
\phi_x^{H}(t) = \arctan\frac{\tilde{x}(t)}{x(t)}
\label{instPhase}
\end{equation}

An alternate description of the instantaneous phase accounting for
$2\pi$ phase jumps is shown below.

\begin{equation}
\beta_x^{H}(t) = \arctan\frac{\tilde{x}(t)}{x(t)}+2\pi l(t)
\end{equation}

where $\beta$ is the accumulated phase. The function $l(t)$ is
incremented or decremented by one for each clockwise or
counter-clockwise orbit of the reconstructed trajectory in the phase
space or transition from $2\pi$ to $-2\pi$. For explanatory purposes
the simpler equation described by equation~\ref{instPhase} will be
used for future descriptions, and a statistical method will be used
to find
a measure of synchronization.\\

This definition of the instantaneous phase only makes sense when
there is a dominant peak in the power spectrum or when oscillations
are extracted via bandpass
filtering for the frequency of interest~\cite{Rosenblum2004}.\\

By using the wavelet approach the frequency range of interest can be
set without bandpass filtering. Below the complex Morlett wavelet is
defined.
\begin{equation}
\psi(t) =
(e^{i\omega_ct}-e^{\omega_c^2t\sigma^2/2})e^{-t^2/2\sigma^2}
\end{equation}

Convolution with time series yields wavelet coefficients
\begin{equation}
W_l(t) = (\psi\circ x)(t) = \int\psi(t')x(t-t')dt'
\end{equation}

Then the phase can be defined
\begin{equation}
\phi_x^W(t)=\arctan\frac{Im(W(t))}{Re(W(T))}
\end{equation}

As mentioned earlier, the $\pm2\pi$ phase jumps add further
complexity to the measurement of phase difference between the
signals. A common method for over coming this problem based on a
statistical approach using the \emph{index based on circular
variance} as defined below

\begin{equation}
\gamma = |\frac{1}{N}\sum_{j=1}^N e^{i[n\phi_x(t)-m\phi_y(t)]}| =
1-CV
\end{equation}

where $CV$ denotes the circular variance of an angular distribution,
and $\gamma$ is also known as the phase locking value (PLV) or the
mean phase coherence. The integers $n$ and $m$ allow comparison of
different frequency, where $n:m$ is the ratio of the mean frequency
of the observed times series. The values of $\gamma$ fall into the
interval $[0,1]$, where values of zero indicate uncorrelated phase
differences and a phase synchronisation of one shows perfect
synchronisation. This measure is independent of amplitude.\\

%Because the phase distribution is non-uniform there will be an
%over-estimation of phase synchronization. To account for the bias
%the phase index $\gamma$ must be compared to a surrogate data set
%$\gamma^S$ with a maximally uniform distribution.
%\begin{equation}
%\gamma^* = \begin{array}{l l}
%  0 & \quad \gamma<\gamma^S\\
%  \frac{\gamma-\gamma^S}{1-\gamma^S} & \quad \gamma\geq \gamma^S\\ \end{array}
%\end{equation}
%
%The surrogate data $\gamma^S$ is composed of the mean of 10 shuffled
%versions of the phase from one the signals. The phase shuffling has
%been achieved by the following process.\\
%
%\begin{itemize}
%\item The first shuffled signal a reversed copy of the original.
%\item The second shuffled signal is the reversed copy the original
%split in two equal length vectors positions switched, then the new
%vector is reversed.
%\item The third shuffled signal uses the same approach as the second
%signal, expect the original is split into thirds
%\end{itemize}

%\[SurData_1 = \phi_n \quad n=\{N,N-1,...,2,1\}\]
%\[SurData_2 = flip\{\phi_{\eta}, \phi{\xi}\}\]
%\[\eta = \{floor(N/2)+1,floor(N/2)+2,...,N\}\]
%\[\xi = \{1,2,...,N/2\}\]
%\[SurData_3 = flip\{\phi_{\alpha}, \phi_{\beta}, \phi_{\delta}\}\]
%\[\alpha = \{floor(2N/3)+1,floor(2N/3)+2,...,N\}\]
%\[\beta = \{floor(N/3)+1,floor(N/3)+2,...,floor(2N/3)\}\]
%\[\delta = \{1,2,...,floor(N/3)\}\]





\bibliographystyle{plain}
\bibliography{BrainSII_biblography.bib}
\end{document}
